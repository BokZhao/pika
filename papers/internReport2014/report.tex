\begin{document}
\maketitle
%%%%%%%%%%%%%%%%%%%%%%%%%%%%%%%%%%%%%%%%%
% Journal Article
% LaTeX Template
% Version 1.3 (9/9/13)
%
% This template has been downloaded from:
% http://www.LaTeXTemplates.com
%
% Original author:
% Frits Wenneker (http://www.howtotex.com)
%
% License:
% CC BY-NC-SA 3.0 (http://creativecommons.org/licenses/by-nc-sa/3.0/)
%
%%%%%%%%%%%%%%%%%%%%%%%%%%%%%%%%%%%%%%%%%

%----------------------------------------------------------------------------------------
%	PACKAGES AND OTHER DOCUMENT CONFIGURATIONS
%----------------------------------------------------------------------------------------

\documentclass[twoside]{article}

\usepackage[sc]{mathpazo} % Use the Palatino font
\usepackage[T1]{fontenc} % Use 8-bit encoding that has 256 glyphs
\linespread{1.05} % Line spacing - Palatino needs more space between lines
\usepackage{microtype} % Slightly tweak font spacing for aesthetics

\usepackage[hmarginratio=1:1,top=32mm,columnsep=20pt]{geometry} % Document margins
\usepackage{multicol} % Used for the two-column layout of the document
\usepackage[hang, small,labelfont=bf,up,textfont=it,up]{caption} % Custom captions under/above floats in tables or figures
\usepackage{booktabs} % Horizontal rules in tables
\usepackage{float} % Required for tables and figures in the multi-column environment - they need to be placed in specific locations with the [H] (e.g. \begin{table}[H])
\usepackage{wrapfig} %FIGURE wrapping

\usepackage{graphicx}

\usepackage{hyperref} % For hyperlinks in the PDF

\usepackage{lettrine} % The lettrine is the first enlarged letter at the beginning of the text
\usepackage{paralist} % Used for the compactitem environment which makes bullet points with less space between them
\usepackage{mathtools} %For equations
\usepackage{amsmath} %For equations Ref
\usepackage{abstract} % Allows abstract customization
\renewcommand{\abstractnamefont}{\normalfont\bfseries} % Set the "Abstract" text to bold
\renewcommand{\abstracttextfont}{\normalfont\small\itshape} % Set the abstract itself to small italic text

\usepackage{titlesec} % Allows customization of titles
\renewcommand\thesection{\Roman{section}} % Roman numerals for the sections
\renewcommand\thesubsection{\Roman{subsection}} % Roman numerals for subsections
\titleformat{\section}[block]{\large\scshape\centering}{\thesection.}{1em}{} % Change the look of the section titles
\titleformat{\subsection}[block]{\large}{\thesubsection.}{1em}{} % Change the look of the section titles

\usepackage{fancyhdr} % Headers and footers
\pagestyle{fancy} % All pages have headers and footers
\fancyhead{} % Blank out the default header
\fancyfoot{} % Blank out the default footer
\fancyhead[C]{Computational Science Internship $\bullet$ May-August, 2014} % Custom header text
\fancyfoot[RO,LE]{\thepage} % Custom footer text

%----------------------------------------------------------------------------------------
%	TITLE SECTION
%----------------------------------------------------------------------------------------

\title{\vspace{-15mm}\fontsize{24pt}{10pt}\selectfont\textbf{PIKA: A MOOSE Based Application for Modeling Snow Microstructure}} % Article title

\author{
\large
\textsc{Micah Johnson}\thanks{Article content was completed at the Idaho National Laboratory}\\[2mm] % Your name
\normalsize  Boise State University \\ % Your institution
\normalsize \href{mailto:micahjohnson1@u.boisestate.edu}{micahjohnson1@u.boisestate.edu} % Your email address
\vspace{-5mm}
\and
\large
\textsc{Andrew Slaughter}\thanks{Assigned Mentor}\\[2mm] % Your name
\normalsize  Idaho National Laboratory \\ % Your institution
\normalsize \href{mailto:andrewslaughter@inl.gov}{andrewslaughter@inl.gov} % Your email address
\vspace{-5mm}
}
\date{}

%----------------------------------------------------------------------------------------

\begin{document}

\maketitle % Insert title

\thispagestyle{fancy} % All pages have headers and footers

%----------------------------------------------------------------------------------------
%	ABSTRACT
%----------------------------------------------------------------------------------------

\begin{abstract}
The Multi-physics Object Oriented Simulation Environment (MOOSE) offers an opportunity to build fully coupled models that span several length scales \cite{Gaston_2009}. As a part of an initiative to  model an entire cryosphere, PIKA is presented for the first time as a capable snow microstructure evolution simulator. PIKA is a phase-field model solving for pore space evolution in dry snow. The code is validated using an experiment in which an air bubble was migrating through ice. The application's capacity is demonstrated by importing a $\mu$-CT scan of microstructure and applying a temperature gradient to induce phase evolution. 

\end{abstract}

%----------------------------------------------------------------------------------------
%	ARTICLE CONTENTS
%----------------------------------------------------------------------------------------

\begin{multicols}{2} % Two-column layout throughout the main article text
%----------------------------------------------------------------------------------------
%	INTRODUCTION
%----------------------------------------------------------------------------------------
\section{Introduction}

\lettrine[nindent=0em,lines=2]{A}ccurately modeling an entire cryosphere requires coupling together several length scales. The smallest of these, the microstructure of snow, is integral to understanding larger scale problems like snow pack longevity, global temperature feedback, and avalanche potential. However, solving multi-scale problems is not only numerically challenging, but requires more computational resources. These difficulties often delay the original science. The Multi-Physics Object Oriented Simulation Environment (MOOSE), developed at the Idaho National Laboratory, provides a framework in which applications can solve partial differential equations using advanced numerical methods, provides fully coupled solutions across multiple length scales, and is parallel agnostic \cite{Gaston_2009}. This allows researchers to build advanced multi-scale models without being detracted by other sciences required to solve their problem. Such models have already been successfully demonstrated at the Idaho National Laboratory using MOOSE in other applications. Thus we propose an entire cryosphere can be modeled using applications built on MOOSE. The first initiative of this proposal is an application called PIKA which models the microstructural evolution of dry snow.

%----------------------------------------------------------------------------------------
%	METHOD
%----------------------------------------------------------------------------------------
\section{Method}

PIKA is a phase-field model for solidification. In brief, the phase-field method is a common treatment for multiphase problems where properties are phase dependent. The method introduces an extra variable that represents the phase of the medium. This technique is useful because material properties can defined as continuous functions of where phi ranges from -1 (water vapor) to 1 (ice). The two phases are separated by a diffuse interface of some assumed thickness (W). Our model follows the work of \cite{Plapp_2009} with the exception that PIKA takes advantage of the finite element method. 
PIKA currently incorporates a single mode of mass transport through sublimation and thus only models dry snow. The microstructural evolution can be modeled by three equations governing the phase evolution \eqref{eq:phase}, heat transport \eqref{eq:heat} and mass transport \eqref{eq:mass}. The unknown variables are the phase ($\phi$), temperature ($T$), and a dimensionless water vapor concentration ($u$).
 \begin{equation} \label{eq:phase}
	\tau \frac{\partial \phi}{\partial t} = W^2 \nabla^2 \phi +(\phi-\phi^3)+\lambda[u-u_{eq}](1-\phi^2)^2 
\end{equation}
\begin{equation}\label{eq:heat}
	C(\phi)\frac{\partial T}{\partial t} = \nabla \cdot [K(\phi) \nabla T] + \frac{L_{sg}}{2}\frac{\partial \phi}{\partial t}
\end{equation} 
\begin{equation} \label{eq:mass}
	\frac{\partial u}{\partial t} = \nabla \cdot[ D(\phi) \nabla u] - \frac{1}{2}\frac{\partial \phi}{\partial t}
\end{equation}
Where u is defined as the difference between vapor density and the saturated vapor density normalized by the density of ice as shown in \eqref{eq:u}
\begin{equation} \label{eq:u}
	u = \frac{ \rho_{vapor}-\rho_{v.s.} (T_{ref})}{\rho_{ice}}
\end{equation}
Equation \eqref{eq:phase} is a classic phase-field equation that represents the bulk phase with the first two terms. The third term drives the phase change based on the availability of vapor. $\tau$ is a relaxation time and $\lambda$ is a coupling constant. Both coefficients are formulated in terms of the capillary length and the interface kinetic coefficient, which are common phase-field terms. For brevity, the derivation of these terms can be seen in \cite{Plapp_2009}. The other two equations are transport equations with phase dependent source terms. In accordance with phase-field techniques the material properties in  \eqref{eq:heat}, \eqref{eq:mass} are linearly interpolated in \eqref{eq:coefficients}. Notice that \ref{eq:diffusion_coefficient} tends towards zero as $\phi$ $\rightarrow$ 0, thus the water vapor does not diffuse through ice. Which is known as a one-sided model in the phase-field community.
\begin{subequations} \label{eq:coefficients}
\begin{align}
	C(\phi) = C_{ice} \frac{1+\phi}{2} + C_{air} \frac{1-\phi}{2} \\
	K(\phi) = K_{ice} \frac{1+\phi}{2} + K_{air} \frac{1-\phi}{2} \\
	D(\phi) = D_{v} \frac{1-\phi}{2} \label{eq:diffusion_coefficient}
\end{align}
\end{subequations}

%--------------
\subsection{Temporal Scaling}
In modeling ice and water vapor, large disparities are present (e.g. water vapor diffusion and microstructure evolution time scales). \cite{Plapp_2009} recommend treating the evolution as a quasi-steady problem. This assumption allows ice density, diffusion terms, and latent heat to be scaled with little impact on the final solution. The velocity of the phase interface is an important metric in the phase-field method, thus the scaling is applied to preserve the interface velocity. This is accomplished by applying the scaling only to the equations \eqref{eq:phase}, \eqref{eq:heat}, and \eqref{eq:mass} and not the property definitions. The scaling process can be seen in \eqref{eq:scaling} where $5e{-5}< \xi < 1$.
\begin{subequations} \label{eq:scaling}
\begin{align}
	D(\phi) \rightarrow D(\phi) \xi \\ 
	K(\phi) \rightarrow K(\phi)\xi  \\
	L_{sg} \rightarrow L_{sg}\xi  \\
	u \rightarrow \frac{u}{\xi}  \\
	u_{eq} \rightarrow \frac{u_{eq}}{\xi}  \\
	\lambda \rightarrow \lambda \xi
\end{align}
\end{subequations}
Note that \eqref{eq:phase} is unaffected by the scaling since it was applied to $u$ and $\lambda$ in the term that drives the phase, they cancel each other out. The mass transport equation scaling is algebraically manipulated to resemble the scaling on the heat transport equation. Initially the scaling applied to \eqref{eq:mass} is shown in equation \eqref{eq:scaled_mass}. 

\begin{equation}
\frac{\partial \frac{u}{\xi}}{\partial t} = \nabla \cdot[ \xi D(\phi) \nabla \frac{1}{\xi}u] - \frac{ 1}{2}\frac{\partial \phi}{\partial t}
 \label{eq:scaled_mass}
\end{equation}
Since $\xi$ is a constant the scaling on the diffusion term is cancelled. By multiplying \eqref{eq:scaled_mass} by $\xi$ our transport equations are scaled exactly the same as seen in \eqref{eq:scaled_transport}

\begin{subequations} \label{eq:scaled_transport}
\begin{align}
	C(\phi)\frac{\partial T}{\partial t} = \xi\nabla [K(\phi) \nabla T] + \xi\frac{L_{sg}}{2}\frac{\partial \phi}{\partial t}\\
	\frac{\partial u}{\partial t} = \xi\nabla \cdot [D(\phi) \nabla u] - \xi \frac{ 1}{2}\frac{\partial \phi}{\partial t}
\end{align}
\end{subequations}
Equations \eqref{eq:phase} and \eqref{eq:scaled_transport} are the strong forms of the PDE's that PIKA solves.
%--------------
%%%%
%\subsection{Implementation in MOOSE}
%Since MOOSE is a finite element framework, all equations must be in their weak form to be used effectivley. The weak form for finite element is started by gathering all terms on one side, multipy by a test function and integrating. By using Gauss's Divergence theorem and integration by parts, diffusion terms can be reduced to a single derivative. An example of this is shown on the mass transport equation in \eqref{eq:weak_ex}.
%begin{subequations} \label{eq:weak_ex}
%FEM Weak Form:
%\begin{align}
%	\int_\omega \! \frac{\partial u}{\partial t} \xi \ - \xi\nabla \cdot [D(\phi) \nabla u] \xi + \xi \frac{ 1}{2}\frac{\partial \phi}{\partial t} \xi \\
%\end{align}

%----------------------------------------------------------------------------------------
%	RESULTS
%----------------------------------------------------------------------------------------
\section{Results}

\begin{table}[H] 
\caption{PIKA bubble simulation parameters}
\label{table:bubble_stats}
\centering
\begin{tabular}{llr}
\toprule
Parameter & Symbol & Value \\
\midrule
Half Width & $x_{max}$ & $0.0025 m$ \\
Height & $y_{max}$ & $0.0050 m$ \\
Top Temp. & $T_{cold}$ & $264.8 K$ \\
Bottom Temp. & $T_{hot}$ & $267.515 K$ \\
Interface Thickness & $w$ & $1e-5 m$ \\
Temporal Scale & $\xi$ & $1e-4$ \\
\bottomrule
\end{tabular}
\end{table}

\begin{figure}[H] 
\centering
  \includegraphics[width=0.5\textwidth]{2D_bubble_7200s_543dT.png}
  \caption{Simulated bubble migration after 7200 seconds with an applied temperature gradient of 543 K/m. The black ring shows the initial condition of the bubble and the magenta shows the interface of the bubble post migration.}
\label{fig:bubble}
\end{figure}

PIKA has been validated by using the same experiment referenced in \cite{Plapp_2009} which was used to validate their code. The experiment in brief was a cube of single crystal ice with a 1 mm hole drilled through the center. The cube had copper plates attached on the top and bottom which were used to induce a temperature gradient. The gradient allows the hole to migrate and the velocity of the bubble was recorded. The experiment was replicated in PIKA taking advantage of half symmetry to reduce computational time. The problem is solved using three simulations used in series. The first is to allow the problem to relax to a steady state in which the equations decoupled. Otherwise the phase interface would relax and the solution would treat this as an actual phase change. Then the relaxed phase solution is used to initialize the temperature distribution, while still decoupled from $\phi$ and $u$. The two previous simulations provide the initial conditions for $\phi$ and $T$ . Some key parameters used in the simulations are provided in table \ref{table:bubble_stats} . 
Figure \ref{fig:bubble} shows the initial position of the bubble interface and the magenta shows the bubble's interface after simulating migration for 7200 seconds with the temperature shown in kelvin. The interface between the phases was calculated to be 3.89e-9 $m/s$. This is the same order of magnitude of the experiment and \cite{Plapp_2009} calculated the velocity for the same experiment to be ~4e-9.
Additionally PIKA is capable of taking a $\mu$-CT scan of snow microstructure as an initial condition. Figure \ref{fig:snow} shows a demonstration of simulated snow microstructure evolution where a $\mu$-CT scan was used as the initial condition for the phase.
\begin{figure}[H] 
\centering
  \includegraphics[width=0.5\textwidth]{2D_snow_3200s_543dT.png}
  \caption{Demonstration of $\mu$-CT scan of snow as an initial condition and simulated to evolve for 3200 seconds.The black contour shows the initial condition of the pore's interface and the magenta contour shows the snow microstructure post migration.}
\label{fig:snow}
\end{figure}




%----------------------------------------------------------------------------------------
%	DISCUSSION
%----------------------------------------------------------------------------------------
\section{Discussion}
Though we observe good agreement with the results seen in \cite{Plapp_2009}, we have found that after long simulated ($~$10,000s) periods the problem become unstable. The simulated interface begins to thin noticeably. When the interface becomes practically zero, $\phi$ begins to produce values outside of $\phi$=(-1,1) which results in a failure material properties. The source of this issue is not yet known. Once this issue is resolved, PIKA's next steps will incorporate more sophisticated growth models that will allow for more anisotropic growth( e.g. dendrites). The first step towards that will be solving for micro-convection within the pore space. Thus expanding the methods of mass transport. Also PIKA is very sensitive to selections in $W$. Currently one-sided models contains a few of the open problems in the phase-field community. Since the vapor does not diffuse in the ice, the precision of the solve and the mesh refinement can actually generate negative diffusion values. Issues with interpolating diffusion coefficients through the interface are often resolved by incorporating them through tensors instead of scalars. However, in the case of the vanishing diffusion there is not a uniform method to make the vapor diffusion term a tensor and will require further research. Despite the problems encountered, PIKA was built on MOOSE with ease compared to building the code separately.  
%----------------------------------------------------------------------------------------
%	CONCLUSION
%----------------------------------------------------------------------------------------
\section{Conclusion}
The first step towards modeling an entire cryosphere is the MOOSE based application presented in this paper named PIKA. Following the work done by \cite{Plapp_2009}, PIKA solves three partial differential equations, which represent the phase evolution, the heat distribution, and the mass transport. To reduce computational time, a temporal scaling was applied to bridge the gap in time scales. Using these techniques we were able to accurately model bubble migration in ice. Additionally, we showed that is possible to use x-ray microtomographic images of real snow microstructure and apply temperature gradients to induce pore space evolution. PIKA is soon to be open-source code and will be available at www.github.com/idaholab/pika.git.    

%----------------------------------------------------------------------------------------
%	REFERENCE LIST
%----------------------------------------------------------------------------------------
\begin{thebibliography}{50} % Bibliography


\bibitem[Gaston, et al., 2009]{Gaston_2009}
Gaston, D.  Newman, C. Hansen, G. Lebrun-Grandie, D. , (2009).
\newblock MOOSE: A parallel computational framework for coupled systems of nonlinear equations
\newblock {\em Nuclear Engineering and Design}, Vol. 239, Issue 10: 1768-1778 

\bibitem[Kaempfer and Plapp, 2009]{Plapp_2009}
Kaempfer, T.~U. and Plapp, M. (2009).
\newblock Phase-field modeling of dry snow metamorhism
\newblock {\em Physical Review}, 79:031502-1:17.

\end{thebibliography}

%----------------------------------------------------------------------------------------

\end{multicols}
\section{Experiencing an INL Internship}
The internship opportunity I was given at the Idaho National Laboratory proved to be invaluable. The topics I was exposed to at the Idaho National Laboratory were numerous. Among them I have gained valuable skills in C++, software development, finite element methods, phase-field methods, and more. I was mentored by Andrew Slaughter who did a great job at pushing me without overworking me. I was also in-part mentored by other members of INL namely the MOOSE and MARMOT development teams. My time there allowed me to demonstrate other skills I had already accumulated in math and programming. It was truly a great experience and I would highly recommend anybody who has the opportunity to take it.

\end{document}
