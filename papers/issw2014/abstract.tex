Using the open-source Multiphysics Object Oriented Simulation Environment (MOOSE; \url{www.mooseframework.org}) from Idaho National Laboratory (INL) the genesis of a modular, collaborative, and multi-scale set of simulation tools for snow was developed with the primary objective of demonstrating the capabilities of the MOOSE framework.

Two independent applications were created: a meso-scale continuum model and a micro-structure model. The continuum model, named Ibex, solves the transient heat equation and accounts for short-wave and long-wave irradiance as well as latent and sensible heat exchange. The micro-structure model, named Pika, was developed following the work of \citet{kaempfer2009phase} and is a fully-coupled 3D finite element, phase field model capable of tracking the phase transition and capturing the heat and mass transfer at the micro-structure scale in the ice matrix and pore space.

The key feature of the models developed is that each was developed using MOOSE and therefore is inherently parallel and expandable, allowing for model expansion including coupling of additional physics (e.g., solid mechanics) and development of multi-scale simulations. Any application developed with MOOSE supports running, in parallel, any other MOOSE-based application. Each can be developed independently, but still easily communicate with one another (e.g., conductivity in the meso-scale model Ibex could be a constant input just as easily as a complete micro-structure Pika model evaluation) without additional code being written.  These two models were then coupled into a single multi-scale simulation, named Yeti.

This method of development has proven effective at INL and the work presented herein aims to be the beginning of a truly collaborative snow modeling effort that greatly increases our current ability to develop sophisticated and sustainable simulation tools.

\bigskip
\noindent KEYWORDS: modeling, micro-structure, multi-scale, vapor diffusion, phase-change
