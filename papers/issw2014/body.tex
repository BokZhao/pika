\section{INTRODUCTION}
Modeling the thermal behavior of snow is not a new endeavor; \citet{lachapelle1960} cites a paper from 1892 that examined temperature profiles of snow. A significant amount of work has examined snow using a continuum mechanics theory of mixtures (e.g., \citet{adams1989, brown1999}.  Using a thermal non-equilibrium approach, \citet{bartelt2004} indicated that temperature differences between the pore air and ice particles and inter-facial heat exchange between snow crystals played a significant role in determining the temperature profile. Perhaps the most comprehensive model developed to date is the SNOWPACK model \citep{lehning1999, bartelt2002, lehning2002a, lehning2002b} that accounts for heat transfer, water transport, vapor diffusion, and mechanical deformation.  Research conducted in an attempt to validate the SNOWPACK model yielded reasonable results, yet \citet{fierz2001} encouraged additional work regarding the initial stage of snow metamorphism, specifically the processes involving particles changing to small faceted or rounded crystals. \citet{miller2003} and \citet{miller2009} provided a unique approach for modeling this transition. They were able to develop a model capable of faceted growth, but the model is limited in a number of ways, including an assumed spherical geometry.

Recent approaches to modeling the snowpack are based on the 3-D images of the snow micro-structure.  One notable article by \citet{kaempfer2005} utilized X-ray micro-tomography ($\mu$-CT) to build a 3-D image of a snow sample to which a finite element model was applied for modeling the heat transfer through the sample. \citet{kaempfer2009phase} demonstrated that phase-field methods may be applied to snow metamorphism and concludes that with the model ``snow metamorphism can be studied in details not possible heretofore.''

Given this broad range modeling approaches a new paradigm is needed for modeling and simulation that fosters rapid development and collaboration. The open-source Multiphysics Object Oriented Simulation Environment (MOOSE; \url{www.moooseframework.org}) is a framework specifically designed for such tasks. MOOSE is a finite-element framework that aids in application development by harnessing state-of-the-art fully-coupled, fully-implicit multiphysics solvers while providing automatic parallelization, mesh adaptivity, and an ever expanding set of physics modules including solid mechanics, phase-field, Navier-Stokes, and heat conduction. These features have provided a recipe for rapid model development (NEED REF). MOOSE natively supports multi-scale models allowing to couple MOOSE-based applications, thus fostering collaborations (NEED REF). Finally, MOOSE follows a rigorous and development strategy that ensures software quality at both the framework and application level (REF WSSSPE).

This paper briefly demonstrates the capabilities of MOOSE by:
\begin{enumerate}
\item Developing a snow micro-structure model, named Pika, based on the work \citep{kaempfer2009phase},
\item Developing a meso-scale continuum model for heat-condition, named Ibex, and
\item Coupling the two models together into a single, multi-scale simulation named Yeti.
\end{enumerate}

The purpose of the work is not to provide a new model for snow, but to provided the basis for a completely new approach to modeling snow, an approach that will bring groups together to build a myriad of modeling tools that utilize a common framework.

\section{PIKA: MICRO-STRUCTURE MODEL}\label{sec:pika}
Methods and results here ...

\section{IBEX: MESO-SCALE MODEL}\label{sec:ibex}
Methods and results here ...

\section{YETI: MULTI-SCALE MODEL}\label{sec:yeti}


\section{CLOSING REMARKS}
